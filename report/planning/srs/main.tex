\documentclass[a4paper]{srs}

\hypersetup{%
	,pdftitle={RNA-Seqlyze SRS}%
	 pdfauthor={Patrick Pfeifer}%
	,pdfkeywords={thesis, rna-seq, srs}%
}

\begin{document}
\begin{titlepage}
\begin{center}
%
% Logo & Project Name
\includegraphics[width=0.4\textwidth]{../../assets/fhnw_hls_e_10mm}
\\[1cm]
{ \large Bachelor Thesis }
%
% Titel
\\[1cm]
{ \sffamily\Huge RNA-Seqlyze }
\\[1cm]
{ \sffamily\LARGE \bfseries Software Requirement Specification }
\\[2cm]
%
% Authors & Clients
\begin{minipage}{0.4\textwidth}
\begin{flushleft}
\large\emph{Student:}\\
	Patrick Pfeifer\\
\end{flushleft}
\end{minipage}
\begin{minipage}{0.4\textwidth}
\begin{flushright}
\large\emph{Professor:}\\
	Prof.~Dr.~Georg Lipps
\end{flushright}
\end{minipage}
%
% Verticall Fill
\vfill
%
% TODO'S
\listoftodos\vfill
%
% Version history
\setlength{\aboverulesep}{0pt}
\setlength{\belowrulesep}{0pt}
\setlength{\extrarowheight}{.75ex}
\begin{tabularx}{\textwidth}{|p{1.2cm}|>{\raggedright}X|X|p{3cm}|}
\rowcolor[gray]{0.8}
	Version & Author & Comment & Date\\
\midrule
	  0.1
	& Patrick Pfeifer
	& Initial draft version
	& 23. Mai 2012
\\
	  \ 
	& \ 
	& \ 
	& \ 
\\
\bottomrule
\end{tabularx}
\end{center}
\end{titlepage}

\tableofcontents
\newpage

	\section{Introduction}

\subsection{Purpose}
This document details the software requirements specification for the RNA-Seqlyze free software project. It defines the intended use of the software and lists the covered use cases.

\subsection{Scope}

The project is started by Patrick Pfeifer as part of his bachelor thesis at the University of Applied Sciences and Arts Northwestern 
Switzerland FHNW at the School of Life Sciences.

RNA-seq data is of increasing importance to the analysis of prokaryotes. The intended use for the software is to aid researchers in anaylzing data generated by next-generation RNA sequencing methods (RNA-seq). Specifically, the software shall help to improve the annotation of existing genome data.

\subsection{asdasd}

Dieses Dokument fungiert als Grundlage für die Abnahme und Validierung der Software nach deren Fertigstellung. Änderungen am Inhalt dieses Dokuments und damit an den Vorgaben für die Software unterliegen einem formellen Prozess. Hierbei wird unter Berücksichtigung terminlicher, technischer und kommerzieller Aspekte über Zustimmung oder Ablehnung der Änderungen abgestimmt. Nach der Abstimmung wird sowohl das Ergebnis wie auch allfällige Änderungen dokumentiert.


\subsection{References}
\begin{itemize}
\item \href{http://standards.ieee.org/findstds/standard/830-1998.html}{
	IEEE Recommended Practice for Software Requirements Specifications
	(IEEE Std 830-1998, Revision of IEEE Std 830-1993)}
\end{itemize}

\subsection{Overview}
Im ersten Kapitel wird die grundlegende Idee kurz erläutert. In Kapitel 2 folgt eine allgemeine Beschreibung der definierten Grundaufgaben und Rahmenbedingungen des Projekts. Daraufhin werden in Kapitel 3 funktionale Anforderungen besprochen und in Form von Use Cases dargestellt. Nach dieser Darstellung wird in Kapitel 4 auf nicht-funktionale Anforderungen eingegangen.

	\section{General Description}

\subsection{Product Environment}
In Krankenhäuser und Arzt-Praxen werden Bilder von Untersuchungen (CT, MRI, PET, Ultraschall, etc.) an Patienten typischerweise auf einem zentralen PACS (Picutre Archiving and Communication System) Server gespeichert. Die Bilder werden in DICOM-Format gespeichert und können von verschiedensten Clients mit Hilfe medizinischer Imaging-Software abgerufen werden. Oft haben verschiedene Ärzte spezielle Bedürfnisse. Ganz allgemein jedoch sollten die Bilder auf dem PACS Server jederzeit und unverzüglich gefunden und angezeigt werden können. Um eine optimale Nutzung der archivierten Bildern zu gewährleisten, möchten die Ärzte die Bilder ausserdem nicht nur am Arbeitsplatz, sondern auch Unterwegs anzeigen und bearbeiten können. Das Anhängen von Tags und Kommentaren an ein Bild erleichtert ihnen die Suche nach spezifische Bildern.

\subsection{Produktfunktionen}
Es soll eine Webanwendung für Ärzte und Dozenten erstellt werden, welche Bilder vom PACS System abrufen wollen. 
Die Webanwendung soll dem Anwender eine Übersicht über alle auf dem PACS Server abgelegten Bilder geben. Ausserdem sollen die zu den Bildern gehörenden Tags, Beschreibung und Kommentaren nach Stichworten oder Wortbruchstücken durchsucht werden können. Die Suchresultate sollen in Form von einer Thumbnail-Galerie angezeigt werden. Ein einzelnes Bild soll auch in voller Grösse inklusive zugehöriger Metadaten angezeigt werden können. Dabei sollen zu jedem Bild Tags und Kommentare hinzugefügt oder gelöscht werden können.

\subsection{User Characteristics}
In diesem Abschnitt werden Benutzer, die mit der Webanwendung interagieren mit Schwerpunkt auf ihre Fähigkeiten im Bereich der Informatik charakterisiert.
	\subsubsection{Researcher:}
Es wird angenommen, dass Ärzte oder auch Assistenzärzte grundlegende Fähigkeiten im Bereich der Informatik besitzen. Sie sollen intuitiv mit der Webanwendung interagieren und sich an gewohnten Abläufen orientieren können.
	\subsubsection{Professor:}
dito

\subsection{Aufgaben und Ziele der Benutzer}

\subsection{Allgemeine Restriktionen}
Die Webanwendung muss mit den Safari Browser kompatibel sein.
Das System wird mit zeitgemässer Hardware betrieben.
Als Programmiersprache wird auf dem Applikationsserver (Backend) Java verwertet und im Browser (Frontend) HTML, CSS und JavaScript (mit jQuery). Der Persistenz-Layer wird vorwiegend mit MySQL Datenbanken gehandhabt.

\subsection{Annahmen und Abhängigkeiten}
Bewährt sich die Software und besteht ein Bedürfnis, so würden in zukünftigen Projekten Software (Äpp's) für SmartPhones und Tablets entwickelt werden. Dieses Projekt soll demnach als Basis für die Entwicklung solchen Äpp's dienen.

	\section{Functional Requirements}

\subsection{Use Cases}
* Product Benefit, ** Implementation Difficulty, *** Priority

\begin{tabularx}{\textwidth}{
	>{\raggedright}p{0.5cm}
	>{\raggedright}p{2.4cm}
	>{\raggedright}X
	l
	l
	l
}
	  Nr
	& Purpose
	& Usage Scenario
	& *
	& **
	& ***
\\\toprule
	  1
	& Alle Bilder anzeigen
	& Der Benutzer kann die Bilder vom PACS Server abrufen.
	  Diese werden in kleinen Formaten angezeigt.
	& hoch & tief & hoch
\\\midrule
	  2
	& Bilder suchen
	& Der Benutzer kann die Beschreibungen, Kommentare und Tags der Bilder nach Stichworten
	  oder Wortbruchstücken durchsuchen ("Volltextsuche").
	& hoch & mittel & hoch
\\\bottomrule
\end{tabularx}

\subsection{Priorities}
The priorites are assigned by the client and have the following meanings.

\begin{tabularx}{\textwidth}{l|X}
	Priority & Meaning
\\\hline
	  high
	& This requirement is indispensible and absolutely necessary for the correct functioning of the product. It must be implemented.
\\\hline
	  medium
	& This requirement is not indispensible but makes a substantial contribution to the usability of the product. It should be implemented.
\\\hline
	  low
	& This requirement contributes towards a better usability of the product but it is not a strictly necessety. The functionality would be nice to have.
\end{tabularx}

	\section{Non-Functional Requirements}

\subsection{Software Quality Requirements}
\subsubsection{Security}
The data transmitted to the application for processing is \emph{not} expected to be privacy-sensitive or confidential. Data submitted by individual users will generally not be presented to other users, but no efforts are made to protect the data from being retreived by other authorized users of the server.

\subsubsection{Usability}
The application's user interface shall be easy to use and follow current conventions. When using the software, the user shall constantly be informed what the current state of the application is, which steps have been already carried out and which ones are to be carried out next.

\subsubsection{Reliability and Maintainability}
An error-free execution of all functions of the software shall be achieved by writing tests for all parts (unit tests) as well as for the whole system (integration tests). The coverage of the unit tests shall reach at least 80 Percent of the source code.

In case the server hosting the application is rebooted, the application shall be automatically restarted as well.

\subsubsection{Performance and Efficiency}
Load times and data transmission shall be within reasonable bounds. Except for the transmission of big data files from the users pc to the application server, there shall be no significant delays when using the application.

The time required to process a given dataset shall be estimated and the estimate shall be presented to the user before he triggers the processing by clicking the designated button.

\subsubsection{Documentation}
The source code shall be documented, such that modifications or additional modules could be integrated by third-party developpers.
Furthermore, an extensive user manual shall be composed.

	\section{Appendices}

\subsection{Glossary}

\begin{tabularx}{\textwidth}{l|X}
	  JavaScript
	& The Scripting Language was developped in the 90s by Brenden Eich
	and has been first published in 1996 as part of the Netscape 2.0 web-browser.
	Today the language is standardised by ECMA-262 and ISO/IEC 16262.

	\href{http://www.ecma-international.org/publications/standards/Ecma-262.htm}{
	Standard ECMA-262; ECMAScript Language Specification; Edition 5.1 (June 2011)}
\\
\end{tabularx}

\end{document}
