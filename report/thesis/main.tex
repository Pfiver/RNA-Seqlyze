\documentclass[a4paper]{thesis}

%\setcounter{tocdepth}{2}
%\setcounter{secnumdepth}{2}
\hypersetup{pdfauthor={Patrick Pfeifer}}

\begin{document}
\begin{titlepage}
\begin{center}
%
% Logo & Project Name
\includegraphics[width=0.4\textwidth]{../../assets/fhnw_hls_e_10mm}
\\[1cm]
{ \large Bachelor Thesis }
%
% Titel
\\[1cm]
{ \sffamily\Huge RNA-Seqlyze }
\\[1cm]
{ \sffamily\LARGE \bfseries Software Requirement Specification }
\\[2cm]
%
% Authors & Clients
\begin{minipage}{0.4\textwidth}
\begin{flushleft}
\large\emph{Student:}\\
	Patrick Pfeifer\\
\end{flushleft}
\end{minipage}
\begin{minipage}{0.4\textwidth}
\begin{flushright}
\large\emph{Professor:}\\
	Prof.~Dr.~Georg Lipps
\end{flushright}
\end{minipage}
%
% Verticall Fill
\vfill
%
% TODO'S
\listoftodos\vfill
%
% Version history
\setlength{\aboverulesep}{0pt}
\setlength{\belowrulesep}{0pt}
\setlength{\extrarowheight}{.75ex}
\begin{tabularx}{\textwidth}{|p{1.2cm}|>{\raggedright}X|X|p{3cm}|}
\rowcolor[gray]{0.8}
	Version & Author & Comment & Date\\
\midrule
	  0.1
	& Patrick Pfeifer
	& Initial draft version
	& 23. Mai 2012
\\
	  \ 
	& \ 
	& \ 
	& \ 
\\
\bottomrule
\end{tabularx}
\end{center}
\end{titlepage}

\tableofcontents
\newpage

\section{Einleitung}
%        =========

\subsection{Motivation / Ziel}
%           -----------------

\subsection{Theoretischer Hintergrund}
%           -------------------------

\section{Architektur der implementierten Lösung}
%        ======================================

\subsection{Komponenten}
%           -----------

\subsection{Technologie-Umfeld}
%           ------------------

\subsection{Begründung von Design und Technologieauswahl}
%           --------------------------------------------

\subsection{Erfahrungsbericht}
%           -----------------

\section{Die Software}
%        ============

\subsection{Screenshots}
%           -----------

\subsection{Quellcode}
%           ---------

\section{Ausblick}
%        ========

\subsection{Komplettierung}
%           --------------

\subsection{Inbetriebnahme}
%           --------------

\section{Anhang}
%        ======

\subsection{Glossar}
%           -------

\begin{tabularx}{\textwidth}{lX}
	  RNA-seq
	& Next-generation Sequenzierungs-Technologe angewendet auf
          das Profiling kompletter Transkriptome
\\
\end{tabularx}

\end{document}

% vim: tw=80:fo+=w
